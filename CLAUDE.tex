\documentclass{article}
\usepackage{xcolor}
\usepackage[margin=2cm]{geometry}
\usepackage{amsmath,amssymb}
\usepackage{fontspec}
\newfontfamily\emojifont{Apple Color Emoji}[Renderer=Harfbuzz,Scale=0.85]
\directlua{
  luaotfload.add_fallback("emojifallback", {
    "AppleColorEmoji:mode=harf;";
  })
}
\setmainfont{Charter}[RawFeature={fallback=emojifallback}]
\directlua{
  local function lower_emoji(head)
    for n in node.traverse(head) do
      if n.id == node.id("glyph") then
        local c = n.char
        if (c >= 0x1F300 and c <= 0x1F9FF) or c == 0x26A0 or c == 0x2705 or c == 0x23F8 or c == 0x2696 or c == 0x2192 then
          n.yoffset = -65536
        end
      end
    end
    return head
  end
  luatexbase.add_to_callback("pre_linebreak_filter", lower_emoji, "lower_emoji")
  luatexbase.add_to_callback("hpack_filter", lower_emoji, "lower_emoji")
}
\usepackage{longtable,booktabs,array}
\usepackage{calc}
\usepackage{graphicx}
\usepackage[colorlinks=true,allcolors=black]{hyperref}
\usepackage[normalem]{ulem}
\definecolor{linkgray}{gray}{0.3}
\let\oldhref\href
\renewcommand{\href}[2]{\oldhref{#1}{\textcolor{linkgray}{\uline{#2}}}}
\usepackage{bigfoot}
\DeclareNewFootnote[para]{E}
\makeatletter
\expandafter\def\csname @makefnbreak\endcsname{%
  \unskip\linebreak[0]\quad%
}
\makeatother
\setcounter{secnumdepth}{-\maxdimen}
\setlength{\emergencystretch}{3em}
\setlength{\parindent}{0pt}
\setlength{\parskip}{3pt plus 1pt minus 0.5pt}
\providecommand{\tightlist}{\setlength{\itemsep}{0pt}\setlength{\parskip}{0pt}}
\AtBeginDocument{%
  \setlength{\topsep}{2pt}%
  \setlength{\partopsep}{0pt}%
  \setlength{\itemsep}{3pt}%
}
\newcounter{none}

\begin{document}

\section{Projekt: Anerkennungsbescheide-Assistenz für
Stephan}\label{projekt-anerkennungsbescheide-assistenz-fuxfcr-stephan}

\subsection{Kontext}\label{kontext}

Wissenschaftlicher Mitarbeiter an Hochschule, bearbeitet
\textasciitilde75 Anerkennungsanträge/Semester. Studis wollen
Prüfungsleistungen aus anderen Studiengängen anerkennen lassen.

\subsection{Aktueller manueller
Workflow}\label{aktueller-manueller-workflow}

\begin{enumerate}
\def\labelenumi{\arabic{enumi}.}
\tightlist
\item
  Antrag kommt im Funktionspostfach ein (\textasciitilde10 Module pro
  Antrag)
\item
  Stephan routet an zuständige Modulverantwortliche (nutzt
  Excel-Zuordnungstabelle)
\item
  Antworten von Profs einsammeln + nachfassen
\item
  Antworten zusammenfassen → Bescheid schreiben (Word/PDF)
\item
  Review (Stephan schaut drüber)
\item
  Email an Studi
\end{enumerate}

\subsection{Was bereits existiert}\label{was-bereits-existiert}

\subsubsection{TH Lübeck Prototyp (analysiert in
/tmp/recog-ai-demo)}\label{th-luxfcbeck-prototyp-analysiert-in-tmprecog-ai-demo}

\textbf{Tech Stack:} Flask + ChromaDB (Vector-DB) + ChatGPT API (über
OpenAI-kompatible Schnittstelle)

\textbf{Workflow:} 1. User: PDF Copy-Paste in Webform (max 10.000
Zeichen) 2. ChatGPT parst externes Modul → JSON (Titel, Credits,
Lernziele, Level, Prüfungsform) 3. ChromaDB: Semantic similarity search
→ Top 5 interne Module 4. User wählt internes Modul → ChatGPT vergleicht
1:1 5. Output: HTML side-by-side Vergleich + Empfehlung

\textbf{Vergleichskriterien (gleichwertig, außer Arbeitsaufwand):}

\begin{itemize}
\tightlist
\item
  Lernziele (primär): ≥80\% → vollständig, ≥50\% → teilweise,
  \textless50\% → keine Anerkennung
\item
  Credits: Externes ≥ Internes OK; Internes \textgreater\textgreater{}
  Externes → max. teilweise
\item
  Bildungsniveau: Bachelor vs.~Master muss passen
\item
  Prüfungsform: Nur wenn für beide vergleichbar
\item
  Arbeitsaufwand: Nur wenn gut vergleichbar (sonst ignoriert)
\end{itemize}

\textbf{Output:} Side-by-side Modul-Vergleich
(Prüfungsamt-Review-tauglich) + begründete Empfehlung

\textbf{Was es NICHT löst:}

\begin{itemize}
\tightlist
\item
  ❌ Keine finale Entscheidung (nur Empfehlung, Profs entscheiden)
\item
  ❌ Kein Workflow-Management (Routing, Status-Tracking, Nachfassen)
\item
  ❌ Kein rechtlich bindender Bescheid-Generator
\item
  ❌ Keine Historisierung
\end{itemize}

\subsubsection{Stephans Datengrundlage}\label{stephans-datengrundlage}

\begin{itemize}
\tightlist
\item
  \textbf{Modulhandbuch als JSON} (✅ bereits konvertiert/in Arbeit)
\item
  \textbf{Excel-Zuordnungstabelle} (Ground Truth): Modul →
  Modulverantwortlicher

  \begin{itemize}
  \tightlist
  \item
    Warum nötig: Modulhandbuch veraltet + Besonderheiten bei
    Anerkennungen
  \item
    Wird in JSON übernommen oder als separater Lookup genutzt
  \end{itemize}
\item
  \textbf{Umrechnungsregeln} (in Excel)
\end{itemize}

\subsection{Ziel: Assistenz-Workflow (NICHT
Vollautomatisierung)}\label{ziel-assistenz-workflow-nicht-vollautomatisierung}

\textbf{✅ Support gewünscht:}

\begin{itemize}
\tightlist
\item
  Routing-Vorschläge: ``Modul X → Prof Müller zuständig''
\item
  Status-Dashboard: ``Prof Schmidt antwortet seit 5 Tagen nicht''
\item
  Bescheid-Entwurf basierend auf Prof-Antworten
\item
  Konsistenz-Check: ``Modul Y wurde letztes Semester schon anerkannt''
\end{itemize}

\textbf{❌ NICHT gewünscht:}

\begin{itemize}
\tightlist
\item
  Automatische Bescheide ohne Review
\item
  Automatische Entscheidungen (bleiben bei Profs)
\item
  Auto-Emails ohne Freigabe
\end{itemize}

\textbf{Idealer Workflow:} 1. Antrag → System schlägt Routing vor →
Stephan bestätigt/korrigiert 2. System trackt Status → Dashboard zeigt
offene Antworten 3. Antworten kommen → System generiert Bescheid-Entwurf
→ Stephan reviewed 4. Stephan gibt frei → System versendet

\subsection{Constraints}\label{constraints}

\begin{itemize}
\tightlist
\item
  User: Kein Programmierer, Windows, M365 verfügbar
\item
  Output: Professionelle rechtliche Dokumente (Briefkopf, sauberes
  Layout)
\item
  ID-System für Anerkennungen innerhalb eines Antrags erforderlich
\item
  Review-Step zwingend erforderlich (Stephan hat finales Wort)
\end{itemize}

\subsection{Technische Komponenten}\label{technische-komponenten}

\subsubsection{Setup-Phase (einmalig)}\label{setup-phase-einmalig}

\begin{enumerate}
\def\labelenumi{\arabic{enumi}.}
\tightlist
\item
  Stephans Modulhandbuch-JSON in Vector-DB laden
\item
  Excel-Zuordnung Modul → Verantwortlicher integrieren
\item
  Optional: Historische Anerkennungen importieren
\end{enumerate}

\subsubsection{Laufender Betrieb}\label{laufender-betrieb}

\begin{itemize}
\tightlist
\item
  TH Lübeck System als Modul-Matching-Engine (oder eigene Instanz)
\item
  Workflow-Layer für Koordination (zu definieren)
\item
  Bescheid-Generator (Word/PDF Template)
\end{itemize}

\subsection{Workflow-Vorschlag (von
Jonathan)}\label{workflow-vorschlag-von-jonathan}

\begin{enumerate}
\def\labelenumi{\arabic{enumi}.}
\tightlist
\item
  Email von Studi → Tool erkennt automatisch passende Module, schlägt
  vor
\item
  Stephan wählt passendes Modul aus → Email an zuständigen Prof (Bitte
  um Prüfung)
\item
  Prof bewertet (manuell ODER mit Tool-Unterstützung) → antwortet
  Stephan
\item
  Tool generiert Bescheid (Zusage/Ablehnung) → Email an Studi
\end{enumerate}

\textbf{Offene Punkte:}

\begin{itemize}
\tightlist
\item
  \textbf{Multi-Modul-Handling:} \textasciitilde10 Module/Antrag → 10
  separate Prof-Emails oder Batch? Übersicht bei 75 Anträge × 10 = 750
  Einzelprüfungen?
\item
  \textbf{Status-Tracking:} Dashboard für ``wer hat nicht geantwortet''?
  Nachfass-Reminder nach X Tagen?
\item
  \textbf{Review vor Versand:} Automatischer Versand an Studi ODER
  Stephan gibt frei?
\item
  \textbf{Prof-Zugang:} Brauchen Profs Login/Tool-Zugang oder
  Email-Antwort ``Ja/Nein + Begründung'' ausreichend?
\item
  \textbf{ID-System:} Identifikation ``Antrag \#123, Modul 3/10'' o.ä.?
\end{itemize}

\subsection{Lösungsansätze (Evaluation
ausstehend)}\label{luxf6sungsansuxe4tze-evaluation-ausstehend}

\begin{enumerate}
\def\labelenumi{\arabic{enumi}.}
\tightlist
\item
  \textbf{Power Automate + Word-Template} - Uni-Infrastruktur, keine
  externen Tools
\item
  \textbf{n8n/Make + Airtable + Documint} - Flexibler, externes Tooling
\item
  \textbf{Custom Flask App} (basierend auf TH Prototyp) +
  Workflow-Extension
\end{enumerate}

\subsection{Offene Fragen}\label{offene-fragen}

\begin{itemize}
\tightlist
\item
  Hat Stephan im JSON bereits Feld \texttt{verantwortlicher}, oder muss
  Excel-Lookup separat bleiben?
\item
  Welche Email-Infrastruktur? (Outlook/Exchange via M365?)
\item
  Wie sehen Anträge konkret aus? (Format, Struktur)
\item
  Bestehende Word-Vorlagen für Bescheide?
\end{itemize}

\subsection{Referenzen}\label{referenzen}

\begin{itemize}
\tightlist
\item
  TH Lübeck Prototyp:
  https://www.unidigital.news/th-luebeck-entwickelt-prototypen-fuer-ki-unterstuetzte-anerkennungsprozesse/
\item
  GitHub Prototype: https://github.com/pascalhuerten/recog-ai-demo
  (analysiert in /tmp/recog-ai-demo)
\end{itemize}

\end{document}
